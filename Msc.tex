\documentclass[12pt,a4paper]{article}
\usepackage{amsmath, amsfonts, amssymb, fancyhdr, floatflt, epsfig, graphicx,psfrag}
\usepackage[latin1]{inputenc}

\usepackage{newtxtext,newtxmath}
\usepackage{multicol}
\usepackage{xcolor}

%A4 = 210x297 => 170x257 with 2cm margins
%=>these are generous ways of implementing this
\textwidth=16cm
\textheight=24.5cm

%now adjust TeX margins to generously fit this 
\topmargin=-0.3in
\oddsidemargin=0.0cm
\evensidemargin=0.0cm

\usepackage{hyperref}
%\hypersetup{
%    colorlinks=true,
%    linkcolor=blue,
%    filecolor=magenta,      
%    urlcolor=cyan,
%}

\newcommand{\ds}{{\sf DarkSUSY}}


\begin{document}
\pagestyle{plain}


 \vspace*{-1.5cm}

\thispagestyle{empty}

{\begin{center}
MSc project \\[1ex]
 {\large \bf  Relic Density of Kaluza-Klein Dark Matter}\\[0.5ex]
 {\scriptsize \bf Jan Fredik Aasmundtveit}
\end{center}
}
 \vspace*{0.5cm}

\noindent{\bf Introduction}\\[-2ex]
 
 There is firm evidence that dark matter (DM) is about five times as prevalent as ordinary 
 matter. This is  inferred in various independent ways, such as from rotational  velocity curves in 
 galaxies, gravitational lensing, velocity distribution of galaxies within galaxy clusters
 or, most importantly, the cosmic microwave background~\cite{Ade:2013zuv}. While most
 likely a new elementary particle \cite{Bertone:2010zza}, the nature of DM is still a matter of debate: 
 no direct or indirect detection experiment has yet reported any uncontroversial DM signal
beyond the purely gravitational evidence.
One of the most promising classes of DM candidates are weakly interacting massive particles (WIMPs), appearing in 
many theories 
aiming to address unrelated problems in the standard model (SM) of particle physics. WIMPs would be thermally 
produced from the
heat bath of SM particles in the early universe, a process which naturally explains the observed DM abundance if 
those particles have masses and coupling strengths at the electroweak scale.
An example for such a WIMP DM candidate arises in the context of  universal extra dimensions (UED),
where the cosmologically observed DM can be associated to the first so-called Kaluza-Klein excitation of the photon
\cite{Servant:2002aq}.


 
\vspace*{0.6cm}
\noindent
{\bf Scope}\\[-2ex]

The FORTRAN package \ds\ \cite{ds} has been developed to numerically calculate 
properties of DM, as well as to make 
detailed predictions for various experiments.
These can then be compared with real particle and astrophysical measurements in order to test the 
model. 

The main goal of the MSc project will be to expand upon a UED module to the \ds\ package, by including more processes as well as extending it to non-minimal models. %more details probably
The focus will be on the (computer-aided) analytical calculation of all relevant annihilation cross sections  
for these DM candidates. At high temperatures, i.e.~early cosmological times,
chemical equilibrium with the heat bath is maintained by DM annihilation and the inverse process of DM
creation from the heat bath. The point at which these processes stop, known
as chemical freeze-out, then sets the relic density of DM as measured today. 

Implementing more (co-)annihilation processes in \ds and extending the module to non-minimal models, Will make it possible to determine  the 
parameter space of this model that leads to a DM density in concordance with astronomical observations can be determined with higher precision.

This project is particularly timely because of very recent updates for the theoretical prediction of 
masses and couplings in this model \cite{Freitas:2017afm}. When implemented in \ds, this thus allows
to significantly improve on previous analyses of the relic density of Kaluza-Klein DM \cite{KKoh2}. 
 
%\newpage 
\vspace*{0.6cm}
\noindent
{\bf Project tasks}\\
{\scriptsize (dates mark deadlines for the completion of the respective task)}\\[-3ex]

\begin{itemize}
\item Familiarize with the standard way of calculating the relic density, including co-annihilations,
as well as with the basic construction of the model of universal extra dimensions {\scriptsize \bf [\emph{July 2019}]}

\item Get an overview of the work done on the module, as well as which missing processes are most significant. Familiarize with mathematica and the FeynCalc package to calculate 
tree-level cross sections for $2\to2$ processes.  {\scriptsize \bf [\emph{August 2019}]}

\item Calculation of as many (co-)annihilation cross sections as possible within the deadline, 
in the stated order of priority {\scriptsize \bf [\emph{September 2019}]}

\item Familiarization with programming in FORTRAN, as well as with \ds and the existing UED module. Devise method for how to
best expand the UED model to non-minimal scenarios{\scriptsize \bf [\emph{October 2019}]}

\item Implement {\it and test}, by comparison of the relic density with previous literature, the processes
calculated until now. 
 {\scriptsize \bf [\emph{November 2019}]}

\item Prepare a short written report of results obtained so far, including the theoretical tools necessary 
to derive them. {\scriptsize \bf [\emph{December 2019}]}
\item Continue to implement missing processes in \ds. {\scriptsize \bf [\emph{February 2020}]}
\item Study parameter ranges that result in correct relic density {\scriptsize \bf [\emph{March 2020}]}
\item Fully focus on thesis writing {\scriptsize \bf [\emph{from 15 March 2020}]}
\end{itemize}


% \vspace*{0.3cm}
%\noindent
%{\bf Methods}
% \vspace*{0.1cm}

%The first part  of the project largely consists in literature research and the compilation of a 
%short written report. The following steps then require mostly analytical calculations, 
%but likely numerical evaluations of the final results (both to be done with the  {\sf mathematica} package).

 \vspace*{0.5cm}
{
\footnotesize
\noindent
{\bf References}

%\renewcommand{\refname}{\footnotesize{References}}
\renewcommand{\refname}{\vspace*{-0.6cm}}

\begin{thebibliography}{99}
\vspace*{-1cm}

\bibitem{Ade:2013zuv} 
  P.~A.~R.~Ade {\it et al.} [Planck Collaboration],
  %``Planck 2013 results. XVI. Cosmological parameters,''
  Astron.\ Astrophys.\  {\bf 571}, A16 (2014)
  [arXiv:1303.5076].
  %%CITATION = ARXIV:1303.5076;%%
  %4038 citations counted in INSPIRE as of 15 Nov 2015
\vspace*{-0.2cm}
  
\bibitem{Bertone:2010zza} 
  G.~Bertone {\it et al.},
  %``Particle Dark Matter: Observations, Models and Searches,''
  Cambridge, UK: Univ. Pr. (2010) 738 p
\vspace*{-0.2cm}

\bibitem{Servant:2002aq} 
  G.~Servant and T.~M.~P.~Tait,
  %``Is the lightest Kaluza-Klein particle a viable dark matter candidate?,''
  Nucl.\ Phys.\ B {\bf 650}, 391 (2003)
%  doi:10.1016/S0550-3213(02)01012-X
  [hep-ph/0206071].
\vspace*{-0.2cm}


\bibitem{ds}
\href{http://www.darksusy.org/}{http://www.darksusy.org/}
\vspace*{-0.2cm}

\bibitem{dsnew}
T.~Bringmann, J.~Edsj\"o, P.~Gondolo, P.~Ullio and L.~Bergstr\"om, 
{DarkSUSY 6.0}, \emph{in prep.} (2016).
\vspace*{-0.2cm}


\bibitem{Freitas:2017afm} 
  A.~Freitas, K.~Kong and D.~Wiegand,
  %``Radiative corrections to masses and couplings in Universal Extra Dimensions,''
  arXiv:1711.07526 [hep-ph].
\vspace*{-0.2cm}

\bibitem{KKoh2}
  M.~Kakizaki, S.~Matsumoto and M.~Senami,
  %``Relic abundance of dark matter in the minimal universal extra dimension model,''
  Phys.\ Rev.\ D {\bf 74}, 023504 (2006)
%  doi:10.1103/PhysRevD.74.023504
  [hep-ph/0605280];
  M.~Kakizaki, S.~Matsumoto, Y.~Sato and M.~Senami,
  %``Significant effects of second KK particles on LKP dark matter physics,''
  Phys.\ Rev.\ D {\bf 71}, 123522 (2005)
%  doi:10.1103/PhysRevD.71.123522
  [hep-ph/0502059];
  M.~Kakizaki, S.~Matsumoto, Y.~Sato and M.~Senami,
  %``Relic abundance of LKP dark matter in UED model including effects of second KK resonances,''
  Nucl.\ Phys.\ B {\bf 735}, 84 (2006)
%  doi:10.1016/j.nuclphysb.2005.11.022
  [hep-ph/0508283].
\vspace*{-0.2cm}


\bibitem{gg} 
  P.~Gondolo and G.~Gelmini,
  %``Cosmic abundances of stable particles: Improved analysis,''
  Nucl.\ Phys.\ B {\bf 360}, 145 (1991).
%  doi:10.1016/0550-3213(91)90438-4
\vspace*{-0.2cm}



\end{thebibliography}


\end{document}